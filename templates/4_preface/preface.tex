\chapter*{Вступ}
\addcontentsline{toc}{chapter}{Вступ}

\textbf{Актуальність роботи} визначається розширенням доведень часткових випадків Теореми Шпернера для мультимножинного випадку, та великого переліку можливих застосувань теореми.

\textit{Об’єкт дослідження} --- теорема німецького математика, яка дає оцінку потужності сімейства невкладених одна в одну множин, яку узагальнюємо на більш широкий спектр математичних об'єктів.

\textbf{Мета дослідження} полягає в тому, щоб отримати можливі шляхи доведення ускладненого випадку теореми для мультимножини в цілому(тобто для будь якої мультимножини). Для цього було розглянуто велику кількість часткових випадків, для того щоб можливо побачити якусь закономірність, або отримати ідеї для подальшого розвитку запропонованого доказу. Наступна мета, дослідити можливості застосування теореми. Було запропоновано три прикладні задачі, у яких теорема відіграє важливу роль: "Теорія Голосувань", "Поділ на сильні та слабкі групи", "Балансування системи графів".

Завдання наступні:
\begin{enumerate}
  \item
    Дослідити можливості прикладного застосування ускладненого випадку теореми.
  \item
    Розробити варіанти доведення ускладненого випадку теореми, на більш загальну кількість частинних випадків, або ж для мультимножини в цілому.
   \item
   	Довести теорему для частинних випадків, які можуть бути використані у приклданих задачах.
   \item
   	Показати можливості застосування теореми для частинних випадків теореми.
   \item
    Побудувати, або запропонувати можливості комбінаторної оцінки потужності сімейства невкладених одна в одну мультипідмножин.
    \item
     Дослідити системи, які можуть бути представлені у вигляді дводольного графу та потребують балансування.
\end{enumerate}

\textbf{Практичне значення одержаних результатів} полягають в тому, що було отримано доведення теореми для нових частинних випадків, які можуть бути використані, як вже було сказано, у переліку прикладних задач. Прикладні задачі для частинних випадків надають можлиість використати математичні об'єкти які задовольняють умовам теореми, тобто бути певним що верхня оцінка потужності сімейства невкладених одна в одну мультимножин обмежена числом $C_n^{n/2}$.


\chapter*{Перелік умовних позначень}
\addcontentsline{toc}{chapter}{Перелік умовних позначень}

У цьому розділі буде наведено необхідний перелік умовних позначень, необхідних для тлумачення матеріалів:
\begin{center}
\begin{tabular}{ |c|c| } 
 \hline
 $a^n$ & кратність входження елемента у множину \\ 
 $\cup$ & об'єднання множин \\ 
 $\cap$  & перетин множин \\ 
 $\overline{deg},\underline{deg}$  & ступінь входження ребер у вершину графа \\ 
 \hline
\end{tabular}
\end{center}



