\chapter*{Висновки}
\addcontentsline{toc}{chapter}{Висновки}

В результаті виконання роботи вдалося довести услкднений випадок теореми для деякої кількості частинних випадків мультимножини. Тобто отримано математичні об'єкти, які задовольняють умовам теореми та використано ці об'єкти для того, щоб показати прикладні можливості застосування теореми. Розглянуто прикладні можливості застосування ускладненого випадку теореми у таких прикладних задачах: "Теорія Голосувань", "Сильні та слабкі групи", "Балансування дводольного графу". Запропоновано можливість статистично обрахувати потужність сімейства невключних одна в одну множин. Приведено рекурентну формулу за допомогою якої можна ітеративно обрахувати кількість різних мультимножин розміру $k$ вибраних із множини потужністю $n$. Розроблено метод доведення ускладненого випадку теореми, який базується на доведенні звичайного випадку теореми.
