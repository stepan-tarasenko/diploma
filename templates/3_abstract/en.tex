\chapter*{Abstract}

\MakeUppercase{Sperner`s theorem, voting theory, balancing the graph system, strong and weak subsets of the multiset}

Qualification work contains: 60 pages, 38 figures, 3 sources.

	In the course of this work the analysis of the proofs of the simple case of the Sperner's Theorem. There were proposed some main and interesting ways to apply theorem using different partial cases that were proved. The main and the most interesting way to apply theorem was the Voting Theory. In the main scope of the Voting theory families of pairwise disjoint sets, so the estimation of the power of such family was distinguished for partial cases. The next application is balancing the system that could be represented as bipartite graph. Relying on partial cases that were proved, the balancing of such systems could be made. The last but not least, proved partial cases could be used for making strong and weak groups that could be used for different cryptographic problems and also for the Voting Theory.
	
	The way to prove the main theorem of this work was announced and used in my bachelor's work. This method uses the prove of the simple case of the theorem and offers the way to reuse that prove for harder cases. The main algorithm for solving the system of linear equations was the Gauss method, because it is very simple and very easy to program, but any algorithm can be used. Also, the multiset theory was described for better understanding. 

	There were proved a lot of different partial cases of multisets that are satisfy the conditions of the theorem. Also some of the research described here was announced on the conference that took place on the 5th of November 2021 named "Modeling, control and information technologies: Proceedings of V International scientific and practical conference".